\documentclass[tocstyle=ref-genre]{ees}

\shorttitle{Opus patheticum}

\begin{document}

\eesTitlePage

\eesCriticalReport{
  – & –  & –    & Grace notes are generally written out in \B2. \\
    &    & vl   & Inconsistent articulation has been tacitly emended. \\
    &    & org  & Bass figures are missing throughout \B2. \\
  \midrule
  1 & 15 & org  & 1st \eighthNote\ in \B2: d8 \\
    & 26 & org  & 1st \eighthNote\ in \B1 and \B2: \flat e8 \\
    & 65 & vla  & bar in \B2: 16 × d′16 \\
  \midrule
  2 & 50 & org  & 4th to 6th \eighthNote\ in \B2: \flat e8–d8–c8 \\
    & 75 & org  & bar in \B1: \flat e2–\crotchetRest \\
    & 79 & org  & bar in \B1: \flat b4–\flat B4–\crotchetRest \\
    & 88 & org  & 2nd \quarterNote\ in \B1 and \B2: \crotchetRest \\
  \midrule
  3 & –  & trb  & trb only appear in \B3 and are designated
                  \textit{ad libitum} on the title page. \\
    & 1–14 & –  & These bars do not appear in \B1, but only in \B2
                  (with the remark “NB dieſe 14 Takte können auch als Ritornell
                  zum Fugis Maria benützt werden.”) and \B3 (where they
                  serve as introduction). \\
    & 13 & vl 2 & last \eighthNote\ in \B3: a′8 \\
  \midrule
  4 & 5  & vl 2 & last \eighthNote\ in \B1: g′8 \\
    & 9–16 & vla & bars missing in \B1 \\
    & 40 & vla  & 3rd \quarterNote\ in \B1 and \B2: \flat a4 \\
    & 40–41 & org & bass figures missing in \B1 \\
    & 85 & vl 1 & 2nd/3rd \eighthNote\ in \B1: f″8–g″16–f″16 \\
    & 102 & org & bar in \B2 one octave higher \\
  \midrule
  5 & –  & –    & This edition replicates the movement from \B1.
                  In this source, bars 159–203 are indicated by the directive
                  “Ritornello da capo al segno fine” (or similar) in vl 1,
                  vl 2, vla, vlne, and org; B contains a multimeasure rest
                  comprising 44 bars. Hence, bars 159–203 correspond to
                  bars 1–45. \B2 differs in two respects from \B1. First,
                  the da capo is shorter, since it only comprises bars 8–11,
                  31–40, and 45 (segno bar between bars 7 and 8;
                  “vi – de” indications in red pencil, skipping bars 12–30
                  and 41–44). Second, bars 59–64 and 119–121 are written in
                  f~minor and \flat e~minor, respectively (here indicated
                  by parenthesized signs). \B3 differs from \B1 in the
                  following respects: (1) The introduction as well as the
                  da~capo omit bars 12–30 and 41–44. (2) Bars 59–64 and 119–121
                  appear in minor keys (like in \B2). (3) The lyrics are
                  the same as in the first movement (i.\,e., “Septem tuos
                  dolores …”). Accordingly, minor differences appear in the
                  B~notes. Here, the lower B~staff corresponds to the B~part
                  of \B3. (4) There are minor differences regarding
                  articulation in all instruments. \\
    & 29 & vl 1 & 3rd \quarterNote\ in \B1: g″8–f″8 \\
    & 85 & B    & grace note missing in \B1 and \B2 \\
    & 129 & vl 2 & 3rd \quarterNote\ in \B1: \flat e′4 \\
    & 132 & vl 2 & bar in \B1 and \B2: d′2. \\
    & 187 & vl 1 & see bar 29 \\
}

\eesToc{
  \begin{movement}{septem}
    \voice[Coro]
    Septem tuos dolores,\\
    Maria, dum considero,\\
    tecum pati labores,\\
    tecum mori desidero.
  \end{movement}

  \begin{movement}{annae}
    \voice[Tenore]
    Annae gemit senectus,\\
    et Simeon affligitur,\\
    doloris ense pectus,\\
    quando tibi transfigitur.
  \end{movement}

  \begin{movement}{fugis}
    \voice[Coro]
    Fugis, Maria, prolis\\
    onusta caro pondere,\\
    et luminare solis\\
    de nocte vis abscondere.
  \end{movement}

  \begin{movement}{aegyptus}
    \voice[Alto]
    Aegyptus est asyli\\
    lucus tuis amoribus,\\
    iuxta fluenta Nyli\\
    liquescis in doloribus.
  \end{movement}

  \begin{movement}{inpatris}
    \voice[Basso]
    In Patris esse rebus\\
    cum nesciebas Filium\\
    visa es tribus diebus\\
    pallere sicut lilium.
  \end{movement}

}

\eesScore

\end{document}
